%*******10********20********30********40********50********60********70********80

\chap{Introduction}

PDF about goals of the project: \\
https://fronter.com/uit/links/files.phtml/5922e2fd6daceef.1280844757$825881100$/Arkiv/inf2900-2017-project.pdf

- the aim(s) or goal(s) of the project;\\
- the intended audience or''beneficiaries'' of the work done;\\
- the scope of the project;\\
- the approach used in carrying out the project;\\
- assumptions on which the work is based\\
- a broad summary of important outcomes

\section{Aim of the project}

Thinking on actually student problems, we tried to find a solution for some of them. We have developed a whole  online platform for any group of people be able to share a calendar. Our aim was to provide a tool to organize  different schedules, being able to see each schedule separately or as a whole calendar with all events together.\\
Our first target was students, but the platform can be used by everyone who need an organiser. The only thing a person need is to create an account and he or she will be able to create a timetable and share it with people to join it.\\
People who joined a timetable will be able to see the whole calendar but, obviously, will not be able to edit it. Doing this, we have keep the safety of a timetable. If someone does not have got an account, no content in the website can be seen, in order to maintain the confidentiality of our users.\\
With the aim of helping new users to create accounts and learn how to use the application, we have completely filled the Help and About pages. Also, just in case some question is not answered in these pages, an email account have been created for contacting us.\\

\section{Aim of the curse}
One of the aims of this project was to learn how to work together in a work environment: share code, deal with different schedules, find a common idea of what to do and how to it. We have learnt that meeting once per week is unavoidable for sharing ideas, problems and struggles, asking for help and be updated.\\
Another aim was to learn to use the framework Ruby on Rails. We found that it is such a powerful web developer, hard to use for the first time, but very sensitive to any change. That is, with so few code lines you can make big changes in you application. Also, comparing with others frameworks, it is quite easy to put together the logic with the graphics: Ruby helps you as much as a software can.