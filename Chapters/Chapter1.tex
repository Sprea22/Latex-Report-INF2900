%*******10********20********30********40********50********60********70********80

\chap{Introduction}
\label{Introduction}
\section{Aim of the project}
\vspace{-5mm}
One of the most difficult thing in the student's life is to properly plan many meetings, lessons, sport trainings... in particular when this activties are shared with other people.
For this reason the aim of this project is to implement an online scheduler.
A scheduler is an instrument that organizes or maintains schedules. Is possible to define a schedule like a list of planned activities or things to be done showing the date and the time.\\
The initial idea was to provide this online service in order to help students with daily time organization issues, but it could actually  be used also in different contexts, such as team trainings, friends meeting and public events.

The implementation of this project will try to reach the following goals:
\vspace{-5mm}
\begin{itemize}
 \setlength{\itemsep}{-5pt}
 \item Let the people be able to create their own calendar and fill it with events.
 \item Allow people to share their own calendar with others.
 \item Let the people be able to check out and have news about other people's calendar.
\end{itemize}
\section{Aim of the course}
\vspace{-5mm}
The implementation of this project has a really important background goal, that is learn how to achieve a group project using an agile development approach.\\
Since the development uses the web application framework Ruby on Rails an another important goals are to discover its potential and utilities, in order to learn how to use it in a proper way.