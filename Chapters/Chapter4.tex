\chap{Team Work}

\section{Agile implementation}
\vspace{-5mm}
As mentioned, Agile is based around incremental iterations, and this project was \\developed throughout three iterations. During each iteration, the Agile development process described in section 2.1.2 has been followed in the best way possible, learning from errors and trying to improve the process from one iteration to the next.

\subsection{First iteration}
\vspace{-5mm}
We started off by deciding which direction we wanted to go; the aim of the application. It was also decided that the first iteration should be focused around the general structure of the platform: static pages (home, about, help, etc.) and basic functionality. \\\\
During the second group meeting we made sketches of the different pages and how we wanted them to look, behave and interact with each other. Also, we planned the basic structure of the schema for the models (see Figure \ref{fig: Models}).\\\\
After design the basic structure, we created several user stories for this iteration, and stored them in a shared document, via Google Drive. Some examples of the stories are:
\vspace{-5mm}
\begin{itemize} \setlength{\itemsep}{-5pt}
\item As a public user, I'll be able to reach the home page. 1 point
\item As a public user, I'll need to be able to create a private user. 5 points
\item As an admin, I'll be able to see the list of users. 5 points
\end{itemize}
\pagebreak
This iteration was the most difficult one, code-wise, due to the relatively steep learning curve of Ruby on Rails. None of us had any prior experience with this framework. For the first few weeks we learned a tremendous amount about Ruby on Rails by reading, and coding along with the very extensive tutorial by Michael Hartl\cite{hartl}.\\\\
It took some time to get used to the framework, and maby we were a bit too enthusiastic, so we ended the first iteration completing around 80\% of the planned user stories.

\subsection{Second iteration}
\vspace{-5mm}
With the basic structure of the application implemented, we planned the second iteration with a bit more modesty than the first. This time, we tried to be more effective and organized and started to use \textit{Pivotal Tracker}\cite{pivotal} for user stories. The improvement of using this platform instead of Google Drive is explained in Section \ref{Pivotal_Tracker}.\\\\
The goal of this iteration was to implement the complete functionality of the platform. We re-planned  the structure of the different web pages, introducing a more advanced functionality. For example, we decided to add timetable subscription to the user functionality.\\
Ruby offers a very easy use of foreign keys; the way one is able utilize the same variable throughout different models, is genius. Mixing HTML and Ruby is also an excellent way to make our lives a lot easier.\\
Even though we weren't able to implement all the user stories scheduled, we did almost everything and had to re-plan just a few things for the third and last iteration. In order to plan it properly, we made a document listing all the errors and bugs we stumbled upon during the two first iterations..

\subsection{Third iteration}
\vspace{-5mm}
The work planned for this one was separated in three parts: 
\vspace{-5mm}
\begin{itemize}\setlength{\itemsep}{-5pt}
	\item[1.] Finish the leftover work from iteration two.
	\item[2.] Improve the graphical interface of the website.
	\item[3.] Sorting out bugs.
\end{itemize}
During this iteration, we did "small" agile iterations, due to the fact we were fixing problems and reiterating the code. \\
We finished the iteration with a fully functioning version of the application, and all the planned functionality implemented.
	
	\section{Team cooperation and communication}
	\vspace{-5mm}
	A big part of this project has been to work in a team, promoting cooperation and communication when developing software. This provides an additional challenge when every member of the team has different schedules, courses, and backgrounds. The team quickly agreed that it was important to have frequent meetings whenever time allowed, and planned one team meeting each week. Team communication was achieved with a chat group, helping the team plan meetings, report progress, or ask questions.\\
	Most meetings would follow this basic structure:
	\vspace{-5mm}
	\begin{itemize}
		\setlength{\itemsep}{-5pt}
		\item Helping each other with any problems.
		\item Discussing new ideas.
		\item Planning what to do in the following period.
	\end{itemize} 
	
	
	
	
	\section{Instruments and Applications}
	\vspace{-5mm}
	\subsection{Git}
	\vspace{-5mm}
	Throughout this development process, \textbf{Git} has been an essential instrument.\\
	Git is a version control system, that helps a software development team to manage changes to the source code over time, keeping track of every modification to the code. If a mistake is made, the team can easily revert the changes, or look back at the last working version to see what's creating the fault. In addition, every team member can commit their work in a branch, helping the team to compare both the master branch and individual branches, allowing all work to be merged such that all the pieces fit together.\cite{versionController}\\\\
	During this project, proficiency and consistency in using Git has had major improvements. In the first iteration, merge conflicts were a constant issue, but as the team gained experience in managing the git repository properly, the conflicts dissipated. The main issue here was tracking files that can't be merged (like the database file and Gemfile.lock) due to either being binary files, or being tied to a specific computer. As of the third iteration, different modifications could be done easily without any issues. Having a version control system like Git proved to facilitate the teamwork between multiple developers, while also showing the importance of how you utilize it. 
	
	\subsection{Pivotal Tracker} \label{Pivotal_Tracker}
	\vspace{-5mm}
	As previously mentioned, the team initially wrote user stories in a shared text document. Although this gave a good overview, it proved to be unreliable when organizing the work. The team decided to look for a tool aiding in this, and ended up using \href{https://www.pivotaltracker.com/}{Pivotal Tracker}.\cite{pivotal} This is a web-based tool (also available as a smartphone app), created by Pivotal Software\cite{pivotallabs}, an agile software development consulting firm, specifically designed to be used in agile project management(see Section \ref{Agile}). \\\\This tool facilitates the creation of a project, that can be joined by several invited users, where users can add user stories and set who will work on what user story. This tool lets the project have customized scores for a user story, in which the team used Fibonacci points. In addition, each user story can contain different specifications, or tasks. Users can "choose" a task, and mark it as \textit{started}, giving an overview of features, their progress, and which team member is working on that specific feature.
	
	